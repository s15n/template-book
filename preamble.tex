\usepackage[tmargin=2cm,rmargin=1in,lmargin=1in,margin=0.85in,bmargin=2cm
    ,footskip=35pt,headheight=15pt
    %,showframe%  % showframe only for checking purpose. Remove for production.
]{geometry} % page margins


\usepackage{fancyhdr} % headers and footers

\pagestyle{fancy}

\renewcommand{\chaptermark}[1]{\markboth{#1}{}}
\renewcommand{\sectionmark}[1]{\markright{#1}{}}

\fancypagestyle{fancy}{
    \fancyhf{}
    %
    % Header %
    \fancyhead[LE]{\scshape Kapitel \thechapter. \leftmark}
    \fancyhead[RO]{\scshape \thesection. \rightmark}
    \renewcommand{\headrulewidth}{0.3pt}
    %
    % Footer %
    \fancyfoot[C]{\thepage}
   % \setlength{\footskip}{10cm}%{15pt}
    %\setlength{\footheight}{15cpi}
    \renewcommand{\footrulewidth}{\iffootnote{0pt}{0.3pt}}
}


\fancypagestyle{plain}{
    \fancyhf{}
    \renewcommand{\headrulewidth}{0pt}
    %
    \fancyfoot[C]{\thepage}
    %\setlength{\footskip}{15pt}
    \renewcommand{\footrulewidth}{\iffootnote{0pt}{0.3pt}}
}


\usepackage{titlesec} % format chapter look

\titleformat{\chapter}{\bfseries\huge}{\thechapter.}{0.5em}{}


\newcommand{\xoverline}[1]{\mskip.5\thinmuskip\overline{\mskip-.5\thinmuskip {#1} \mskip-.5\thinmuskip}\mskip.5\thinmuskip} % overline short

\usepackage[utf8]{inputenc}
\usepackage[svgnames]{xcolor}
\usepackage{amsmath, amsthm, amssymb, amsfonts,mathtools}
\usepackage{tikz, tikz-cd}
\usepackage{parskip} % Space between paragraphs
\usepackage{thmtools}
\usepackage{framed} % Bars left of theorems
\usepackage[varbb]{newpxmath} % font
\usepackage{dsfont} % for bb 1
\usepackage{euscript}
\usepackage{xfrac} % \sfrac 1/2
\usepackage[makeroom]{cancel}
\usepackage{bookmark}
\usepackage{enumitem}
\usepackage{hyperref,theoremref}

\usepackage{imakeidx}
\usepackage[totoc]{idxlayout} % package for including index page in the table of content


\hypersetup{%
	pdftitle={\booktitle},%
	colorlinks=true, linkcolor=Blue,
	bookmarksnumbered=true,%
	bookmarksopen=true,%
}
 
%%%%% THEOREMS %%%%%

\newenvironment{Leftbar}[1]{%
    \def\FrameCommand{{\color{#1}\vrule width 3pt} \hspace{10pt}}%
    \MakeFramed {\advance\hsize-\width \FrameRestore}
}{\endMakeFramed}

\newcommand{\NewTheoremStyle}[3][\sffamily]{%
    \declaretheoremstyle[%
        headfont=\sffamily\bfseries,%
        notefont=\sffamily\bfseries,%
        notebraces={}{},%
        headpunct=,%
        bodyfont=#1,%
        headformat=\color{#3}\NAME~\NUMBER\hfill\NOTE\smallskip\linebreak,%
        preheadhook=\begin{Leftbar}{#3},%
        postfoothook=\end{Leftbar},%
    ]{custom#2}
}

\newcommand{\NewTheorem}[3][]{%
    \declaretheorem[style=custom#2, numberwithin=chapter, #1]{#3}
}

\NewTheoremStyle{definition}{DarkOrange}
\NewTheoremStyle{lemma}{DarkCyan}
\NewTheoremStyle{proposition}{DarkSlateBlue}
\NewTheoremStyle{theorem}{Crimson}
\NewTheoremStyle{corollary}{MediumVioletRed}
\NewTheoremStyle[]{example}{OliveDrab}
\NewTheoremStyle[]{remark}{Gray}
\NewTheoremStyle{claim}{SeaGreen}

\NewTheorem[title=Satz]{theorem}{theorem}
\NewTheorem[title=Theorem, numberlike=theorem]{theorem}{theoremKai}
\NewTheorem[numberlike=theorem]{definition}{definition}
\NewTheorem[title=Definition]{definition}{definitionSC}
\NewTheorem[numberlike=theorem]{lemma}{lemma}
\NewTheorem[numberlike=theorem]{proposition}{proposition}
\NewTheorem[title=Korollar, numberlike=theorem]{corollary}{corollary}
\NewTheorem[title=Beispiel, numbered=no]{example}{example}
\NewTheorem[title=Bemerkung, numbered=no]{remark}{remark}
\newcounter{claimC} \NewTheorem[title=Behauptung, sibling=claimC]{claim}{claim}
\newcounter{hlemmaC} \NewTheorem[title=Hilfslemma, sibling=hlemmaC]{lemma}{hlemma}
\NewTheorem[title=Behauptung, numbered=no]{claim}{claim*}

%%%%% AMS Proof Fix (Copied from amsthm docs) %%%%%

\makeatletter
\renewenvironment{proof}[1][\proofname]{\par
\pushQED{\qed}%
\normalfont \topsep6\p@\@plus6\p@\relax
\trivlist
\item\relax
{\itshape
\textbf{#1}\@addpunct{:}}\hspace\labelsep\ignorespaces
}{%
\popQED\endtrivlist\@endpefalse
}
\makeatother

\renewcommand{\proofname}{Beweis}



%%%%% Other %%%%%

\setlength\parindent{0pt} % no indent

\renewcommand{\theenumi}{\bfseries\alph{enumi}}
\renewcommand{\theenumii}{\textit{\roman{enumii}}}

\newcommand*{\listhack}{\leavevmode\vspace{-1.5em}}

\renewcommand*\contentsname{Inhalt}
\renewcommand*\chaptername{Kapitel}
